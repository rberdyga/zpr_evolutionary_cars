\documentclass[10pt]{article}
\usepackage{geometry}                % See geometry.pdf to learn the layout options. There are lots.
\geometry{letterpaper}                   % ... or a4paper or a5paper or ... 
%\geometry{landscape}                % Activate for for rotated page geometry
%\usepackage[parfill]{parskip}    % Activate to begin paragraphs with an empty line rather than an indent

%%%%%%%%%%%%%%%%%%%%
\newcommand{\hide}[1]{}

\usepackage{natbib}
\usepackage{xcolor}
\usepackage{url}
\usepackage{hyperref}
\usepackage{mathtools}
\usepackage{polski}
\usepackage[utf8]{inputenc}

\hide{
\usepackage{amscd}
\usepackage{amsfonts}
\usepackage{amsmath}
\usepackage{amssymb}
\usepackage{amsthm}
\usepackage{cases}		 
\usepackage{cutwin}
\usepackage{enumerate}
\usepackage{epstopdf}
\usepackage{graphicx}
\usepackage{ifthen}
\usepackage{lipsum}
\usepackage{mathrsfs}	
\usepackage{multimedia}
\usepackage{wrapfig}
}
\bibliographystyle{humanbio}

	 
%\input{/usr/local/LATEX/Lee_newcommands.tex}
\newcommand{\itemlist}[1]{\begin{itemize}#1\end{itemize}}
\newcommand{\enumlist}[1]{\begin{enumerate}#1\end{enumerate}}
\newcommand{\desclist}[1]{\begin{description}#1\end{description}}

\newcommand{\Answer}[1]{\begin{quote}{\color{blue}#1}\end{quote}}
\newcommand{\AND}{\wedge}
\newcommand{\OR}{\vee}
\newcommand{\ra}{\rightarrow}
\newcommand{\lra}{\leftrightarrow}

\title {Projekt ZPR: Ewolucja pojazdów}
\author{Rafał Berdyga \\
Marta Mejer \\
}
\date{}
%\date{}                                           % Activate to display a given date or no date

\begin{document}
\maketitle
{\bf Opis zadania}:  Oprogramowanie przeprowadzające ewolucję sztucznych pojazdów w 2D. \\ \par
Pojazdy (populacja N osobników reprezentowanych na planszy przez obiekty graficzne) poruszają się po
prostej drodze. Celem jest przejechanie jak najdłuższego odcinka omijając pojawiające się na jezdni
przeszkody (statyczne i dynamiczne). Pojazdy podlegają ewolucji poprzez krzyżowanie cech najlepszych
osobników poprzedniej generacji i mutację. Użytkownik może wpływać na proces ewolucji poprzez zmienianie wartości parametrów takich jak współczynnik mutacji czy liczebność populacji początkowej.
Program ma postać responsywnej aplikacji okienkowej z prostym GUI. Oprócz planszy, po której
poruszają się pojazdy, w oknie programu znajduje się też menu dla użytkownika, przy pomocy którego
może on ustawiać wartości wybranych parametrów, oraz tabela przedstawiająca klasyfikację
najlepszych pojazdów wygenerowanych podczas jednego procesu ewolucyjnego.
\section*{Lista Funkcjonalności:}
\itemlist{
\item Wyświetlanie planszy wraz poruszającymi się pojazdami oraz przeszkodami (statycznymi i dynamicznymi)
\item Ewolucja generowanych pojazdów
\item Podręczne menu dla użytkownika, w którym może on ustawiać parametry takie jak:
\itemlist{
\item rozmiar populacji
\item stopień mutacji
}
\item Wyświetlanie tabeli z danymi najlepszych pojazdów
}
\clearpage
\section*{Podział prac:}
\maketitle
{\bf Suma}: 108h
\itemlist{
\item Poszerzanie wiedzy w zakresie algorytmów ewolucyjnych - 10h
\item Front-end - 23h
\itemlist{
\item UX - 15h
\itemlist{
\item intuicyjny interfejs użytkownika - 7h
\item responsywność interfejsu i planszy - 8h
}
\item prezentacja obiektów graficznych na planszy - 5h
\item prezentacja danych na temat wyników symulacji - 3h
}
\item Back-end - 65h
\itemlist{
\item algorytm generujący nowe pokolenie pojazdów - 39h
\itemlist{
\item tworzenie pierwszego pokolenia pojazdów o losowych parametrach - 3h
\item tworzenie kolejnych pokoleń pojazdów - 36h
\itemlist{
\item algorytm wybierający pożądane cechy z poprzedniego pokolenia - 15h
\item przekazywanie odpowiednich genotypów do dziedziczących pokoleń - 14h
\item mutacja osobników pokolenia - 7h
}
\item ruch pojedynczego pojazdu - 4h
}
\item generacja przeszkód napotykanych na drodze (wraz z obsługą kolizji) - 16h
\itemlist{
\item przeszkody statyczne - 6h
\item przeszkody poruszające się w przeciwnym kierunku - 10h
}
\item generacja obiektów wspomagających pojazdy ewolucyjne - 10h
}
\item Testowanie oprogramowania - 10h
}

\end{document}  
%%%%%%%%%%%%%%%%%%%%%%%%%%%%%%%%%%%%%%%%%%%%%%